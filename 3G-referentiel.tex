\documentclass[a4paper,12pt,fleqn]{article}
\input{preambule_coop}
\input{preambule-partages}

\theme{geo}{Référentiel}{3\up{e}}{Géométrie}


\titleformat*{\subsection}{\color{couleur_theme}\bfseries}
\begin{document}
\renewcommand{\labelitemi}{}

\subsection*{Homothétie et rotation}

\begin{itemize}
	\item \reference{3G10}Transformer une figure par rotation et comprendre l’effet d’une rotation.
	\item \reference{3G11}Transformer une figure par homothétie et comprendre l’effet d’une homothétie.
	\item \reference{3G12}Identifier des rotations et des homothéties dans des frises, des pavages et des rosaces.
	\item \reference{3G13}Mobiliser les connaissances des figures, des configurations, de la rotation et de l’homothétie pour déterminer des grandeurs géométriques.
	\item \reference{3G14}Calculer des grandeurs géométriques (longueurs, aires et volumes) en utilisant les transformations (symétries, rotations, translations, homothétie).
	

\end{itemize}

\subsection*{Théorème de Thalès}

\begin{itemize}
	\item \reference{3G20}Calculer une longueur avec le théorème de Thalès.
	\item \reference{3G21}Démontrer que des droites sont parallèles avec le théorème de Thalès.
	\item \reference{3G22}Connaître et utiliser une définition et une propriété caractéristique des triangles semblables.
\end{itemize}


\subsection*{Trigonométrie}

\begin{itemize}
	\item \reference{3G30}Calculer une longueur dans un triangle rectangle.
	\item \reference{3G31}Calculer la mesure d'un angle dans un triangle rectangle.
	\item \reference{3G32}Résoudre un problème géométrique.
\end{itemize}

\subsection*{Espace}

\begin{itemize}
	\item \reference{3G40}Se repérer sur une sphère (latitude, longitude).
	\item \reference{3G41}Construire et mettre en relation différentes représentations des solides étudiés au cours du cycle (représentations en perspective cavalière, vues de face, de dessus, en coupe, patrons) et leurs sections planes.
	\item \reference{3G42}Calculer le volume d’une boule.
	\item \reference{3G43}Calculer les volumes d’assemblages de solides étudiés au cours du cycle.
\end{itemize}
	
\end{document}

