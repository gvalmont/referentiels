\documentclass[a4paper,12pt,fleqn]{article}	\input{preambule_coop}
\input{preambule-partages}

\theme{gestion}{Référentiel}{3\up{e}}{Organisation et gestion de données, fonctions}


\titleformat*{\subsection}{\color{couleur_theme}\bfseries}
\begin{document}
\renewcommand{\labelitemi}{}

\subsection*{Généralités sur les fonctions}

\begin{itemize}
	\item \reference{3F10}Utiliser les notations et le vocabulaire fonctionnels.
	\item \reference{3F11}Passer d’un mode de représentation d’une fonction à un autre.
	\item \reference{3F12}Déterminer, à partir de tous les modes de représentation, l’image d’un nombre.
	\item \reference{3F13}Déterminer un antécédent à partir d‘une représentation graphique ou d’un tableau de valeurs d’une fonction.
	\item \reference{3F14}Modéliser un phénomène continu par une fonction.
	\item \reference{3F15}Résoudre des problèmes modélisés par des fonctions en utilisant un ou plusieurs modes de représentation.
\end{itemize}

\subsection*{Fonctions affines et linéaires}

\begin{itemize}
	\item \reference{3F20}Représenter graphiquement une fonction linéaire, une fonction affine.
	\item \reference{3F21}Interpréter les paramètres d’une fonction affine suivant l’allure de sa courbe représentative.
	\item \reference{3F22}Modéliser une situation de proportionnalité à l’aide d’une fonction linéaire.
	\item \reference{3F23}Déterminer de manière algébrique l’antécédent par une fonction, dans des cas se ramenant à la résolution d’une équation du premier degré.
\end{itemize}

\subsection*{Proportionnalité}

\begin{itemize}
	\item \reference{3P10}Utiliser le lien entre pourcentage d’évolution et coefficient multiplicateur.
	\item \reference{3P11}Mener des calculs sur des grandeurs mesurables, notamment des grandeurs composées, et exprimer les résultats dans les unités adaptées.
	\item \reference{3P12}Résoudre des problèmes utilisant les conversions d’unités sur des grandeurs composées.
	\item \reference{3P13}Vérifier la cohérence des résultats du point de vue des unités pour les calculs de grandeurs simples ou composées.
	\item \reference{3P14}Résoudre des problèmes en utilisant la proportionnalité dans le cadre de la géométrie.
\end{itemize}

\subsection*{Statistiques}

\begin{itemize}
	\item \reference{3S10}Lire, interpréter et représenter des données sous forme d’histogrammes pour des classes de même amplitude.
	\item \reference{3S11}Calculer et interpréter l’étendue d’une série présentée sous forme de données brutes, d’un tableau, d’un diagramme en bâtons, d’un diagramme circulaire ou d’un histogramme.
	\item \reference{3S12}Calculer des effectifs et des fréquences.
	\item \reference{3S15}Calculer des étendues.
\end{itemize}

\subsection*{Probabilités}

\begin{itemize}
	\item \reference{3S20}À partir de dénombrements, calculer des probabilités pour des expériences aléatoires simples à une ou deux épreuves.
	\item \reference{3S21}Faire le lien entre stabilisation des fréquences et probabilités.
\end{itemize}
	
\end{document}

