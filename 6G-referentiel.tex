\documentclass[a4paper,12pt,fleqn]{article}		\input{preambule_coop}
\input{preambule-partages}

\theme{geo}{Référentiel}{6\up{e}}{Géométrie}

\doublespacing

\titleformat*{\subsection}{\color{couleur_theme}\bfseries}
\begin{document}
\renewcommand{\labelitemi}{}

\subsection*{Constructions géométriques - Niveau 1}

\begin{itemize}
	\item \reference{6G10}Connaitre le vocabulaire et les notations des points, des droites, des segments, des demi-droites et des cercles. 
	\item \reference{6G11}Tracer des perpendiculaires.
	\item \reference{6G12}Tracer des parallèles.
	\item \reference{6G13}Tracer des rectangles et des carrés de longueurs données.
	\item \reference{6G14}Exécuter un programme de construction de niveau 1.
\end{itemize}

\subsection*{Constructions géométriques - Niveau 2}

\begin{itemize}
	\item \reference{6G20}Connaître le vocabulaire des polygones.
	\item \reference{6G21}Tracer un polygone avec le compas et l’équerre.
	\item \reference{6G22}Connaître le vocabulaire et les notations des angles.
	\item \reference{6G23}Utiliser le rapporteur pour tracer ou mesurer un angle.
	\item \reference{6G24}Tracer le symétrique d’une figure.
	\item \reference{6G25}Tracer la médiatrice d’un segment.
\end{itemize}

\subsection*{Constructions géométriques - Niveau 3}

\begin{itemize}
	\item \reference{6G30}Exécuter un programme de construction complexe.
	\item \reference{6G31}Agrandissement ou réduction de figures.
	\item \reference{6G32}Connaître et utiliser les propriétés de conservation de la symétrie axiale.
	\item \reference{6G33}Connaître et utiliser les propriétés des polygones particuliers.
\end{itemize}
	
\end{document}

