\documentclass[a4paper,12pt,fleqn]{article}	\input{preambule_coop}
\input{preambule-partages}

\theme{nombres}{Référentiel}{3\up{e}}{Nombres et Calculs}


\titleformat*{\subsection}{\color{couleur_theme}\bfseries}
\begin{document}
\renewcommand{\labelitemi}{}

\subsection*{Calcul littéral}

\begin{itemize}
	\item \reference{3L10}Déterminer l’opposé d’une expression littérale.
	\item \reference{3L11}Développer (par simple et double distributivités), factoriser, réduire des expressions algébriques simples.
	\item \reference{3L12}Factoriser une expression du type $a^2-b^2$ et développer des expression du type $(a+b)(a-b)$.
	\item \reference{3L13}Résoudre algébriquement une équation du premier degré.
	\item \reference{3L14}Résoudre algébriquement une équation produit.
	\item \reference{3L15}Résoudre algébriquement une équations de la forme $x^2=a$ sur des exemples simples.
	\item \reference{3L16}Résoudre des problèmes se ramenant à une équation, qui peuvent être internes aux mathématiques ou en lien avec d’autres disciplines.

\end{itemize}

\subsection*{Arithmétique}

\begin{itemize}
	\item \reference{3A10}Trouver les diviseurs et les multiples d'un nombre entier.
	\item \reference{3A11}Décomposer un nombre entier en produit de facteurs premiers (à la main, à l’aide d’un tableur ou d’un logiciel de programmation).
	\item \reference{3A12}Simplifier une fraction pour la rendre irréductible.
	\item \reference{3A13}Modéliser et résoudre des problèmes mettant en jeu la divisibilité (engrenages, conjonction de phénomènes...).
\end{itemize}

\subsection*{Nombres et calculs}

\begin{itemize}
	\item \reference{3N10}Utiliser les puissances d’exposants positifs ou négatifs pour simplifier l’écriture des produits.
	\item \reference{3N11}Calculer avec les nombres rationnels, notamment dans le cadre de résolution de problèmes.
	\item \reference{3N12}Résoudre des problèmes mettant en jeu des racines carrées.
	\item \reference{3N13}Résoudre des problèmes avec des puissances, notamment en utilisant la notation scientifique.
\end{itemize}
	
\end{document}

