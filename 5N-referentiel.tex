\documentclass[a4paper,12pt,fleqn]{article}	
\input{preambule_coop}
\input{preambule-partages}

\theme{nombres}{Référentiel}{5\up{e}}{Nombres et calculs}

\titleformat*{\subsection}{\color{couleur_theme}\bfseries}
\begin{document}
\renewcommand{\labelitemi}{}

\subsection*{Calculs}

\begin{itemize}
	\item \reference{5C10}	Calculer le quotient et le reste dans une division euclidienne.																								
	\item \reference{5C11}	Traduire un enchaînement d’opérations à l’aide d’une expression avec des parenthèses.																								
	\item \reference{5C12}	Effectuer un enchaînement d’opérations en respectant les priorités opératoires.	
\end{itemize}

\subsection*{Arithmétique}

\begin{itemize}																									
	\item \reference{5A10}	Déterminer si un nombre entier est ou n’est pas multiple ou diviseur d’un autre nombre entier.																								
	\item \reference{5A11}	Utiliser les critères de divisibilité (par 2, 3, 5, 9, 10).																								
	\item \reference{5A12}	Déterminer les nombres premiers inférieurs ou égaux à 30.																								
	\item \reference{5A13}	Décomposer un nombre entier strictement positif en produit de facteurs premiers inférieurs à 30.																								
	\item \reference{5A14}	Modéliser et résoudre des problèmes faisant intervenir les notions de multiple, de diviseur, de quotient et de reste.	
\end{itemize}

\subsection*{Numération et fractions}

\begin{itemize}																									
	\item \reference{5N10}	Utiliser les écritures décimales et fractionnaires et passer de l’une à l’autre.																								
	\item \reference{5N11}	Relier fractions, proportions et pourcentages.																								
	\item \reference{5N12}	Décomposer une fraction sous la forme d’une somme (ou d’une différence) d’un entier et d'une fraction																								
	\item \reference{5N13}	Reconnaître et produire des fractions égales.																								
	\item \reference{5N14}	Comparer, ranger, encadrer des fractions dont les dénominateurs sont égaux ou multiples l’un de l'autre.
\end{itemize}

\subsection*{Calculs avec des fractions}

\begin{itemize}							
	\item \reference{5N20}	Additionner ou soustraitre des fractions dont les dénominateurs sont égaux ou multiples l’un de l’autre.																								
	\item \reference{5N21}	Utiliser la décomposition en facteurs premiers inférieurs pour produire des fractions égales.	
\end{itemize}

\subsection*{Relatifs - Niveau 1}

\begin{itemize}																								
	\item \reference{5R10}	Utiliser la notion d'opposé.																								
	\item \reference{5R11}	Repérer un point sur une droite graduée les nombres décimaux relatifs.																								
	\item \reference{5R12}	Repérer un point dans le plan muni d’un repère orthogonal.		
\end{itemize}

\subsection*{Relatifs - Niveau 2}

\begin{itemize}																																								
	\item \reference{5R20}	Additionner des nombres décimaux relatifs.																								
	\item \reference{5R21}	Soustraire des nombres décimaux relatifs.
	\item \reference{5R22}	Effectuer une somme algébrique.
	\item \reference{5R23}	Résoudre des problèmes faisant intervenir des nombres décimaux relatifs et des fractions. %+Contrôler la vraisemblance d'un résultat
\end{itemize}

\subsection*{Calcul littéral}

\begin{itemize}																								
	\item \reference{5L10}	Produire une expression littérale pour élaborer une formule ou traduire un programme de calcul.																								
	\item \reference{5L12}	Utiliser le calcul littéral pour démontrer une propriété générale.																								
	\item \reference{5L13}	Utiliser la distributivité simple pour réduire une expression littérale de la forme $ax + bx$ où $a$ et $b$ sont des nombres décimaux.																								
	\item \reference{5L14}	Calculer la valeur d’une expression littérale.																								
	\item \reference{5L15}	Tester si une égalité où figurent une ou deux indéterminées est vraie quand on leur attribue des valeurs numériques.						
	\item \reference{5L16}	Simplifier l'écriture d'une expression littérale. %(Utiliser les notations 2a pour a × 2 ou 2 × a et ab pour a × b, a² pour a × a et a³ pour a × a × a.)
\end{itemize}

\end{document}