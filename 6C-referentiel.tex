\documentclass[a4paper,12pt,fleqn]{article}	\input{preambule_coop}
\input{preambule-partages}

\theme{nombres}{Référentiel}{6\up{e}}{Calculs}

\doublespacing

\titleformat*{\subsection}{\color{couleur_theme}\bfseries}
\begin{document}
\renewcommand{\labelitemi}{}

\subsection*{Calculs - Niveau 1}

\begin{itemize}
	\item \reference{6C10}Additionner, soustraire et multiplier des nombres entiers.
	\item \reference{6C11}Calculer des divisions euclidiennes simples.
	\item \reference{6C12}Résoudre des problèmes de niveau 1.
	\item \reference{6C13}Multiplier un nombre entier par 0,1 ; 0,5
	\item \reference{6C14}Additionner et soustraire en utilisant la distributivité simple (ajouter 14 revient à ajouter 10 puis 4 ; soustraire 14 revient à soustraire 10 puis 4)
	\item \reference{6C15}Faire le lien entre « la moitié de » et multiplier par 1/2

\end{itemize}

\subsection*{Calculs - Niveau 2}

\begin{itemize}
	\item \reference{6C20}Additionner et soustraire des nombres décimaux.
	\item \reference{6C21}Calculer une division euclidienne de niveau 2.
	\item \reference{6C22}Résoudre des problèmes de niveau 2.
	\item \reference{6C23}Multiplier un nombre décimal par 0,1 ; 0,5
	\item \reference{6C24}Diviser un nombre entier par 0,1 ; 0,5
	\item \reference{6C25}Multiplier en utilisant la distributivité simple (multiplier par 14 revient à multiplier par 10, multiplier par 4, puis faire la somme ; 3x + 7x = (3+7)x = 10x
	\item \reference{6C26}Ajouter et soustraire des fractions décimales de même dénominateur.

\end{itemize}

\subsection*{Calculs - Niveau 3}

\begin{itemize}
	\item \reference{6C30}Multiplier des nombres décimaux.
	\item \reference{6C31}Effectuer une division décimale.
	\item \reference{6C32}Résoudre des problèmes de niveau 3.
	\item \reference{6C33}Diviser un nombre décimal par 0,1 ; 0,5.
	\item \reference{6C34}Organiser un calcul en une seule ligne, utilisant si nécessaire des parenthèses.
	\item \reference{6C35}Utiliser une calculatrice pour introduire la priorité de la multiplication sur l’addition et la soustraction.
	\item \reference{6C36}Ajouter et soustraire des fractions de même dénominateur.

\end{itemize}


	
\end{document}

