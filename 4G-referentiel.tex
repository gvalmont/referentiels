\documentclass[a4paper,12pt,fleqn]{article}	\input{preambule_coop}
\input{preambule-partages}

\theme{geo}{Référentiel}{4\up{e}}{Géométrie}


\titleformat*{\subsection}{\color{couleur_theme}\bfseries}
\begin{document}
\renewcommand{\labelitemi}{}

\subsection*{Translation et rotation}

\begin{itemize}
	\item \reference{4G10}Transformer une figure par translation.
	\item \reference{4G11}Identifier des translations dans des frises et des pavages.
	\item \reference{4G12}Comprendre et utiliser l’effet d’une translation : conservation du parallélisme, des longueurs, des aires et des angles.
	\item \reference{4G15}Tranformations de triangle.*
\end{itemize}

\subsection*{Théorème de Pythagore}

\begin{itemize}
	\item \reference{4G20}Calculer une longueur avec le théorème de Pythagore.
	\item \reference{4G21}Démontrer qu'un triangle est rectangle ou non à l'aide du théorème de Pythagore.
	\item \reference{4G22}Résoudre un problème géométrique en ayant recours au théorème de Pythagore.
	\item \reference{4G23}Utiliser les carrés parfaits de 1 à 144.
	\item \reference{4G24}Connaître la définition de la racine carrée d’un nombre positif.
	\item \reference{4G25}Encadrer la racine carrée d’un nombre positif entre deux entiers.%+Utiliser la calculatrice pour déterminer une valeur approchée de la racine carrée d’un nombre positif.
	\item \reference{4G26}Utiliser la racine carrée d’un nombre positif en lien avec des situations géométriques (théorème de Pythagore ; agrandissement, réduction et aires).
\end{itemize}

\subsection*{Théorème de Thalès}

\begin{itemize}
	\item \reference{4G30}Calculer une longueur avec le théorème de Thalès.
	\item \reference{4G31}Démontrer que des droites sont parallèles avec le théorème de Thalès.
	\item \reference{4G32}Résoudre un problème géométrique en ayant recours aux théorèmes de Thalès et de Pythagore.
	\item \reference{4G33}Connaître et utiliser une définition et une propriété caractéristique des triangles égaux.
\end{itemize}

\subsection*{Cosinus d'un angle}

\begin{itemize}
	\item \reference{4G40}Calculer une longueur avec le cosinus d'un angle.
	\item \reference{4G41}Calculer la mesure d’un angle à partir de son cosinus.*
	\item \reference{4G42}Résoudre un problème géométrique en ayant recours au cosinus d'un angle ou aux théorèmes de Thalès et de Pythagore.
\end{itemize}

\subsection*{Espace}

\begin{itemize}
	\item \reference{4G50}Construire et mettre en relation une représentation en perspective cavalière et un patron d’une pyramide.
	\item \reference{4G51}Construire et mettre en relation une représentation en perspective cavalière et un patron d’un d’un cône de révolution.
	\item \reference{4G52}Se repérer dans un pavé droit et utiliser le vocabulaire du repérage : abscisse, ordonnée, altitude.
	\item \reference{4G53}Calculer le volume d’une pyramide, d’un cône.
\end{itemize}
	
\end{document}

