\documentclass[a4paper,12pt,fleqn]{article}		
\input{preambule_coop}
\input{preambule-partages}

\theme{grandeurs}{Référentiel}{6\up{e}}{Grandeurs et mesures}
\doublespacing

\titleformat*{\subsection}{\color{couleur_theme}\bfseries}
\begin{document}
\renewcommand{\labelitemi}{}

\subsection*{Grandeurs et mesures - Niveau 1}

\begin{itemize}
	\item \reference{M10}Déterminer le périmètre d’un polygone.
	\item \reference{M11}Déterminer l’aire d’un carré ou d’un rectangle (ou de figures composées de rectangles et carrés).
	\item \reference{M12}Convertir des longueurs.
\end{itemize}

\subsection*{Grandeurs et mesures - Niveau 2}

\begin{itemize}
	\item \reference{M20}Déterminer l’aire d’un triangle.
	\item \reference{M21}Déterminer l’aire d’un polygone par assemblage ou par découpage.
	\item \reference{M22}Déterminer le périmètre ou l’aire d’un disque.
	\item \reference{M23}Convertir des aires.
	\item \reference{M24}Résoudre un problème en utilisant les périmètres et les aires.


\end{itemize}

\subsection*{Grandeurs et mesures - Niveau 3}

\begin{itemize}
	\item \reference{M30}Déterminer le volume d’un pavé droit.
	\item \reference{M31}Convertir des volumes (et faire le lien avec les contenances).
\end{itemize}
	
\end{document}

