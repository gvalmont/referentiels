\documentclass[a4paper,12pt,fleqn]{article}	\input{preambule_coop}
\input{preambule-partages}

\theme{grandeurs}{Référentiel}{5\up{e}}{Grandeurs et mesures}


\titleformat*{\subsection}{\color{couleur_theme}\bfseries}
\begin{document}
\renewcommand{\labelitemi}{}

\subsection*{Périmètre et aire}

\begin{itemize}
	\item \reference{5M10}Calculer le périmètre et l’aire des figures usuelles (rectangle, parallélogramme, triangle,disque). %+ exprimer le résultat dans l'unité adaptée +vérifier la cohérence des résultats du point de vue des unités
	\item \reference{5M11}Calculer le périmètre et l’aire d’un assemblage de figures.%+ exprimer le résultat dans l'unité adaptée +vérifier la cohérence des résultats du point de vue des unités
	\item \reference{5M12}Effectuer des conversions d’unités de longueurs.
	\item \reference{5M13}Effectuer des conversions d’unités d’aires.
\end{itemize}

\subsection*{Volume}

\begin{itemize}
	\item \reference{5M20}Calculer le volume d’un pavé droit, d’un prisme droit, d’un cylindre.%+ exprimer le résultat dans l'unité adaptée +vérifier la cohérence des résultats du point de vue des unités
	\item \reference{5M21}Calculer le volume d’un assemblage de pavés, prismes et/ou cylindres.%+ exprimer le résultat dans l'unité adaptée +vérifier la cohérence des résultats du point de vue des unités
	\item \reference{5M22}Effectuer des conversions d’unités de volumes.
	\item \reference{5M23}Utiliser la correspondance entre les unités de volume et de contenance pour effectuer des conversions. %+ exprimer le résultat dans l'unité adaptée +vérifier la cohérence des résultats du point de vue des unités
\end{itemize}

\subsection*{Durée}

\begin{itemize}
	\item \reference{5M30}Effectuer des conversions d’unités de durées.
	\item \reference{5M31}Effectuer des calculs de durées et d’horaires. %+ exprimer le résultat dans l'unité adaptée +vérifier la cohérence des résultats du point de vue des unités
\end{itemize}

	
\end{document}

