\documentclass[a4paper,12pt,fleqn]{article}	\input{preambule_coop}
\input{preambule-partages}

\theme{grandeurs}{Référentiel}{6\up{e}}{Gestion de Données}


\titleformat*{\subsection}{\color{couleur_theme}\bfseries}
\begin{document}
\renewcommand{\labelitemi}{}

\subsection*{Proportionnalité}

\begin{itemize}
	\item \reference{6P10}Reconnaitre des problèmes relevant de la proportionnalité.
	\item \reference{6P11}Résoudre un problème relevant de la proportionnalité avec les propriétés de linéarité.
	\item \reference{6P12}Calculer et utiliser un coefficient de proportionnalité.
	\item \reference{6P13}Appliquer un pourcentage.
	\item \reference{6P14}Reproduire une figure en respectant une échelle donnée.
	\item \reference{6P15}Résoudre un problème impliquant des échelles ou des vitesses.
\end{itemize}

\subsection*{Statistiques}

\begin{itemize}
	\item \reference{6S10}Lire une représentation de données (tableaux ; diagrammes en bâtons, circulaires ou semi-circulaires ; graphiques cartésiens).
	\item \reference{6S11}Organiser les données en vue de les traiter.
\end{itemize}
	
\end{document}

