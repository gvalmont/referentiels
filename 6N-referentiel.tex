\documentclass[a4paper,12pt,fleqn]{article}	
\input{preambule_coop}
\input{preambule-partages}

\theme{nombres}{Référentiel}{6\up{e}}{Numération et fractions}

\doublespacing
\titleformat*{\subsection}{\color{couleur_theme}\bfseries}
\begin{document}
\renewcommand{\labelitemi}{}

\subsection*{Numération et fractions - Niveau 1}

\begin{itemize}
	\item \reference{6N10}Connaitre le système décimal.
	\item \reference{6N11}Comparer, ranger, encadrer, repérer des grands nombres entiers.
	\item \reference{6N12}Multiplier ou diviser un entier par 10, 100, 1~000 \ldots (résultat entier)
	\item \reference{6N13}Utiliser les préfixes multiplicateurs (déca à kilo).*
	\item \reference{6N14}Comprendre et utiliser la notion de fraction dans des cas simples.
	%\item \reference{6N15}Résoudre un problème avec des grands nombres.

\end{itemize}

\subsection*{Numération et fractions - Niveau 2}

\begin{itemize}
	\item \reference{6N20}Faire le lien entre les fractions et les nombres entiers.
	\item \reference{6N21}Repérer et placer des fractions sur une demi-droite graduée (origine visible).*
	\item \reference{6N22}Faire des calculs simples avec des fractions à l’aide d’un dessin.*
	\item \reference{6N23}Comprendre et utiliser différentes écritures d’un nombre.
	\item \reference{6N24}Multiplier ou diviser un entier par 10, 100, 1~000 \ldots (résultat décimal)
\end{itemize}

\subsection*{Numération et fractions - Niveau 3}

\begin{itemize}
	\item \reference{6N30}Repérer et placer des nombres décimaux (jusqu’à 3 décimales) sur une demi-droite graduée adaptée.
	\item \reference{6N31}Comparer, ranger, encadrer, intercaler des nombres décimaux.
	\item \reference{6N32}Repérer et placer des fractions sur une demi-droite graduée (origine non visible).*
	\item \reference{6N33}Calculer la fraction d’une quantité.*
	\item \reference{6N34}Utiliser les préfixes diviseurs (milli à téra).*
	\item \reference{6N35}Multiplier ou diviser un nombre décimal par 10, 100, 1 000…
\end{itemize}

\subsection*{Numération et fractions - Niveau 4}

\begin{itemize}
	\item \reference{6N40}Repérer et placer des nombres décimaux (jusqu’à 4 décimales) sur une demi-droite graduée.
	\item \reference{6N41}Établir des égalités entre des fractions simples.*
	\item \reference{6N42}Multiples et diviseurs des nombres d’usage courant.*
	\item \reference{6N43}Critères de divisibilité.*
	\item \reference{6N44}Utiliser des fractions pour exprimer un quotient. Comprendre que b × a/b = a
\end{itemize}

	
\end{document}

