\documentclass[a4paper,12pt,fleqn]{article}	\input{preambule_coop}
\input{preambule-partages}

\theme{nombres}{Référentiel}{4\up{e}}{Calculs}


\titleformat*{\subsection}{\color{couleur_theme}\bfseries}
\begin{document}
\renewcommand{\labelitemi}{}

\subsection*{Relatifs}

\begin{itemize}
	\item \reference{4C10}Effectuer des produits ou des quotients avec des nombres relatifs.
	\item \reference{4C11}Calculer avec des nombres relatifs.
\end{itemize}

\subsection*{Fractions}

\begin{itemize}
	\item \reference{4C20}Comparer, ranger et encadrer des nombres rationnels (positifs ou négatifs).
	\item \reference{4C21}Additionner ou soustraire des nombres relatifs en écriture fractionnaire.
	\item \reference{4C22}Multiplier ou diviser des nombres relatifs en écriture fractionnaire.
	\item \reference{4C23}Effectuer un calcul avec des nombres relatifs et fractionnaires.
	\item \reference{4C24}Utiliser les nombres premiers pour reconnaître et produire des fractions égales ; pour simplifier des fractions.
	\item \reference{4C25}Résoudre des problèmes avec des nombres rationnels.
\end{itemize}

\subsection*{Puissances}

\begin{itemize}
	\item \reference{4C30}Utiliser les puissances de 10 d’exposants positifs ou négatifs.
	\item \reference{4C31}Utiliser les préfixes de nano à giga.
	\item \reference{4C32}Associer, dans le cas des nombres décimaux, écriture décimale, écriture fractionnaire et notation scientifique.
	\item \reference{4C33}Utiliser les puissances d’exposants strictement positifs d’un nombre pour simplifier l’écriture des produits.
\end{itemize}

\subsection*{Calcul littéral}

\begin{itemize}
	\item \reference{4L10}Utiliser la propriété de distributivité simple pour développer un produit ou réduire une expression littérale.
	\item \reference{4L11}Utiliser la propriété de distributivité simple pour factoriser une somme.
	\item \reference{4L12}Démontrer l’équivalence de deux programmes de calcul.
	\item \reference{4L13}Introduire une lettre pour désigner une valeur inconnue et mettre un problème en équation.
	\item \reference{4L14}Tester si un nombre est solution d’une équation.
	\item \reference{4L15}Résoudre algébriquement une équation du premier degré.
\end{itemize}

	
\end{document}

