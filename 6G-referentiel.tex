\documentclass[a4paper,12pt,fleqn]{article}		\input{preambule_coop}
\input{preambule-partages}

\theme{geo}{Référentiel}{6\up{e}}{Géométrie}

\doublespacing

\titleformat*{\subsection}{\color{couleur_theme}\bfseries}
\begin{document}
\renewcommand{\labelitemi}{}

\subsection*{Constructions géométriques - Niveau 1}

\begin{itemize}
	\item \reference{6G10}Connaitre le vocabulaire et les notations des points, des droites, des segments, des demi-droites et des cercles. 
	\item \reference{6G11}Tracer des perpendiculaires.
	\item \reference{6G12}Tracer des parallèles.
	\item \reference{6G13}Tracer des rectangles et des carrés de longueurs données.
	\item \reference{6G14}Utiliser les propriétés des droites perpendiculaires.
	\item \reference{6G15}Exécuter un programme de construction simple.
	\item \reference{6G16}Coder les triangles et les quadrilatères particuliers.
	\item \reference{6G17}Compléter ou rédiger un programme de construction d’une figure plane.
\end{itemize}

\subsection*{Constructions géométriques - Niveau 2}

\begin{itemize}
	\item \reference{6G20}Connaître le vocabulaire des polygones.
	\item \reference{6G21}Tracer un polygone avec le compas et l’équerre.*
	\item \reference{6G22}Connaître le vocabulaire et les notations des angles. %Estimer si un angle est droit, aigu ou obtus.
	\item \reference{6G23}Utiliser le rapporteur pour tracer ou mesurer un angle.
	\item \reference{6G24}Tracer le symétrique d’une figure. %+ compléter une figure par symétrie axiale
	\item \reference{6G25}Tracer la médiatrice d’un segment en utilisant sa définition ou sa caractérisation.
	\item \reference{6G26}Reconnaître et coder la définition de la médiatrice d’un segment, ainsi que sa caractérisation.
	\item \reference{6G27}Être capable de verbaliser et d’expliciter sa méthode de construction du symétrique d’un point, d’un segment ou d’une droite par rapport à un axe donné.
	\item \reference{6G28}Reconnaître, nommer et décrire des figures complexes (assemblages de figures simples).
	\item \reference{6G29}Réaliser une figure plane simple à l’aide d’un logiciel de géométrie dynamique.
\end{itemize}

\subsection*{Constructions géométriques - Niveau 3}

\begin{itemize}
	\item \reference{6G30}Exécuter un programme de construction complexe.
	%\item \reference{6G31}Agrandissement ou réduction de figures.
	\item \reference{6G32}Connaître et utiliser les propriétés de conservation de la symétrie axiale.
	\item \reference{6G33}Connaître et utiliser les propriétés des polygones particuliers.*
	\item \reference{6G34}Représenter, reproduire, tracer ou construire des figures complexes (assemblages de figures simples).
	\item \reference{6G35}Construire la figure symétrique d’une figure donnée par rapport à un axe donné à l’aide d’un logiciel de géométrie dynamique.
	\item \reference{6G36}Réaliser une figure composée de figures simples à l’aide d’un logiciel de géométrie dynamique.
\end{itemize}

\subsection*{Espace}

\begin{itemize}
	\item \reference{6G40}Reconnaître des solides (pavé droit, cube, cylindre, prisme droit, pyramide, cône, boule) à partir d’un objet réel, d’une image, d’une représentation en perspective cavalière.
	\item \reference{6G41}Construire et mettre en relation une représentation en perspective cavalière et un patron d’un cube ou d’un pavé droit.
	\item \reference{6G42}Connaître et utiliser les propriétés géométriques des cubes et pavés droits.*
	\item \reference{6G43}Reconnaître, nommer et décrire des assemblages de solides simples.
	\item \reference{6G44}Construire une maquette à l’aide de patrons d’un assemblage de solides simples (cube, pavé droit, prisme droit, pyramide) dont les patrons sont donnés pour les prismes et les pyramides.
\end{itemize}

\subsection*{Alignement, perpendicularité et parallélisme}

\begin{itemize}
	\item \reference{6G50}Connaître la définition de l’alignement de 3 points ainsi que de l’appartenance à une droite et savoir reconnaître ces situations.
	\item \reference{6G51}Déterminer le plus court chemin entre un point et une droite. %+ Estimer la distance entre un point et une droite.
	\item \reference{6G52}Connaître les relations entre perpendicularité et parallélisme et savoir s’en servir pour raisonner.
\end{itemize}

\subsection*{Se repérer et se déplacer}

\begin{itemize}
	\item \reference{6G60}Se repérer, décrire (tourner à gauche, à droite ; faire demi-tour ; effectuer un quart de tour à droite, à gauche) ou exécuter des déplacements.
	\item \reference{6G61}Programmer des déplacements absolus d’un robot ou ceux d’un personnage sur un écran. %vers le haut, l’ouest…
	\item \reference{6G62}Programmer des déplacements relatifs d’un robot ou ceux d’un personnage sur un écran. %tourner à sa gauche, à sa droite ; faire demi-tour ; effectuer un quart de tour à sa droite, à sa gauche…
\end{itemize}

\end{document}

