\documentclass[a4paper,12pt,fleqn]{article}	\input{preambule_coop}
\input{preambule-partages}

\theme{gestion}{Référentiel}{4\up{e}}{Gestion de données}



\titleformat*{\subsection}{\color{couleur_theme}\bfseries}
\begin{document}
\renewcommand{\labelitemi}{}

\subsection*{Statistiques}

\begin{itemize}
	\item \reference{4S10}Lire, interpréter et représenter des données sous forme de diagrammes circulaires.
	\item \reference{4S11}Calculer et interpréter la médiane d’une série de données de petit effectif total.
\end{itemize}

\subsection*{Probabilités}

\begin{itemize}
	\item \reference{4S20}Utiliser le vocabulaire des probabilités : expérience aléatoire, issues, événement, probabilité, événement certain, événement impossible, événement contraire.%+ savoir que la probabilité d'un événement est un nombre compris entre 0 et 1.
	\item \reference{4S21}Reconnaître des événements contraires et s’en servir pour calculer des probabilités. %+Exprimer des probabilités sous diverses formes (nombre compris entre 0 et 1, pourcentage, fraction).
	\item \reference{4S22}Calculer des probabilités. %+Exprimer des probabilités sous diverses formes (nombre compris entre 0 et 1, pourcentage, fraction).
\end{itemize}

\subsection*{Proportionnalité}

\begin{itemize}
	\item \reference{4P10}Reconnaître sur un graphique une situation de proportionnalité ou de non proportionnalité.
	\item \reference{4P11}Calculer une quatrième proportionnelle par la procédure de son choix.
	\item \reference{4P12}Utiliser une formule liant deux grandeurs dans une situation de proportionnalité.
	\item \reference{4P13}Résoudre des problèmes en utilisant la proportionnalité dans le cadre de la géométrie.
	\item \reference{4P14}Construire un agrandissement ou une réduction d’une figure donnée.
	\item \reference{4P15}Utiliser un rapport d’agrandissement ou de réduction pour calculer, des longueurs, des aires, des volumes.
	\item \reference{4P16}Effectuer des conversions d’unités sur des grandeurs composées.
\end{itemize}

\subsection*{Notion de fonction}

\begin{itemize}
	\item \reference{4F10}Produire une formule littérale représentant la dépendance de deux grandeurs.
	\item \reference{4F11}Représenter la dépendance de deux grandeurs par un graphique.
	\item \reference{4F12}Utiliser un graphique représentant la dépendance de deux grandeurs pour lire et interpréter différentes valeurs sur l’axe des abscisses ou l’axe des ordonnées.
\end{itemize}
	
\end{document}

