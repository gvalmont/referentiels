\documentclass[a4paper,12pt,fleqn]{article}	\input{preambule_coop}
\input{preambule-partages}

\theme{gestion}{Référentiel}{5\up{e}}{Grandeurs et mesure}


\titleformat*{\subsection}{\color{couleur_theme}\bfseries}
\begin{document}
\renewcommand{\labelitemi}{}

\subsection*{Proportionnalité}

\begin{itemize}
	\item \reference{5P10}Reconnaître une situation de proportionnalité ou de non proportionnalité́ entre deux grandeurs.
	\item \reference{5P11}Résoudre des problèmes de proportionnalité avec des procédures variées (additivité, homogénéité, passage à l’unité, coefficient de proportionnalité).
	\item \reference{5P12}Partager une quantité en deux ou trois parts selon un ratio donné.
	\item \reference{5P13}Utiliser l’échelle d’une carte.
\end{itemize}

\subsection*{Statistiques}

\begin{itemize}
	\item \reference{5S10}Recueillir et organiser des données.
	\item \reference{5S11}Lire et interpréter des données brutes ou présentées sous forme de tableaux, de diagrammes et de graphiques.
	\item \reference{5S12}Représenter, sur papier ou à l’aide d’un tableur-grapheur, des données sous la forme d’un tableau, d’un diagramme ou d’un graphique.
	\item \reference{5S13}Calculer des effectifs et des fréquences.
	\item \reference{5S14}Calculer et interpréter la moyenne d’une série de données.
\end{itemize}

\subsection*{Probabilités}

\begin{itemize}
	\item \reference{5S20}Placer un événement sur une échelle de probabilités.
	\item \reference{5S21}Calculer des probabilités dans des situations simples d’équiprobabilité.
\end{itemize}

	
\end{document}

