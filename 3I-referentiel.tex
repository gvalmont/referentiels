\documentclass[a4paper,12pt,fleqn]{article}
\input{preambule_coop}
\input{preambule-partages}

\theme{algo}{Référentiel}{3\up{e}}{Algorithmique et Programmation}


\titleformat*{\subsection}{\color{couleur_theme}\bfseries}
\begin{document}
\renewcommand{\labelitemi}{}

\subsection*{Niveau 1}

\begin{itemize}
	\item \reference{3I01}Réaliser des activités d’algorithmique débranchée.
	\item \reference{3I02}Mettre en ordre et/ou compléter des blocs fournis par le professeur pour construire unprogramme simple sur un logiciel de programmation.
	\item \reference{3I03}Écrire un script de déplacement ou de construction géométrique utilisant des instructionsconditionnelles et/ou la boucle « Répéter ... fois ».
\end{itemize}
	
\subsection*{Niveau 2}

\begin{itemize}
	\item \reference{3I04}Gérer le déclenchement d'un script en réponse à un événement.
	\item \reference{3I05}Écrire une séquence d’instructions (condition « si ... alors » et boucle « répéter ... fois »).
	\item \reference{3I06}Intégrer une variable dans un programme de déplacement, de construction géométrique oude calcul.
\end{itemize}
	
\subsection*{Niveau 3}

\begin{itemize}
	\item \reference{3I07}Décomposer un problème en sous-problèmes et traduire un sous-problème en créant un« bloc-personnalisé ».
	\item \reference{3I08}Construire une figure en créant un motif et en le reproduisant à l’aide d’une boucle.
	\item \reference{3I09}Utiliser simultanément les boucles « Répéter ... fois » et « Répéter jusqu’à ... » ainsi que lesinstructions conditionnelles pour réaliser des figures, des programmes de calculs, desdéplacements, des simulations d’expérience aléatoire.
	\item \reference{3I10}Écrire plusieurs scripts fonctionnant en parallèle pour gérer des interactions et créer des jeux.
\end{itemize}
	
\end{document}

