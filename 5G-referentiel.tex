\documentclass[a4paper,12pt,fleqn]{article}	\input{preambule_coop}
\input{preambule-partages}

\theme{geo}{Référentiel}{5\up{e}}{Géométrie}


\titleformat*{\subsection}{\color{couleur_theme}\bfseries}
\begin{document}
\renewcommand{\labelitemi}{}

\subsection*{Symétries}

\begin{itemize}
	\item \reference{5G10}Transformer une figure par symétrie axiale.
	\item \reference{5G11}Transformer une figure par symétrie centrale.
	\item \reference{5G12}Identifier des symétries dans des frises, des pavages, des rosaces.
	\item \reference{5G13}Utiliser les propriétés de conservation du parallélisme, des longueurs et des angles.
\end{itemize}

\subsection*{Triangles}

\begin{itemize}
	\item \reference{5G20}Construire des triangles connaissant des longueurs et/ou des angles.
	\item \reference{5G21}Connaître et utiliser l'inégalité triangulaire.
	\item \reference{5G22}Connaître et utiliser la définition de la médiatrice.
	\item \reference{5G23}Connaître et utiliser la définition des hauteurs d’un triangle.
\end{itemize}

\subsection*{Angles}

\begin{itemize}
	\item \reference{5G30}Connaître et utiliser les caractérisations angulaires du parallélisme (angles alternes internes, angles correspondants).
	\item \reference{5G31}Connaître et utiliser la somme des angles d’un triangle.
\end{itemize}

\subsection*{Parallélogrammes}

\begin{itemize}
	\item \reference{5G40}Connaître et construire un parallélogramme.
	\item \reference{5G41}Connaître et construire un parallélogramme particulier.
	\item \reference{5G42}Connapitre et utiliser les propriétés des parallélogrammes.
\end{itemize}

\subsection*{Espace}

\begin{itemize}
	\item \reference{5G50}Reconnaître des solides (pavé droit, cube, cylindre, prisme droit, pyramide, cône, boule) à partir d’un objet réel, d’une image, d’une représentation en perspective cavalière.
	\item \reference{5G51}Construire et mettre en relation une représentation en perspective cavalière et un patron d’un pavé droit, d’un cylindre et d'un prisme droit.
    \item\reference{5G52}Connaître et utiliser les propriétés géométriques des cubes, prismes droits et cylindres.
\end{itemize}
	
\end{document}

